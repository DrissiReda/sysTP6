\documentclass{report}
\usepackage[utf8]{inputenc}
\usepackage[francais]{babel}
\usepackage[T1]{fontenc}
\usepackage{lmodern}
\usepackage{textcomp}
\usepackage{listings}
\usepackage{graphicx}
\usepackage{hyperref}
\usepackage{titlesec}
\usepackage{xcolor}
\usepackage{tcolorbox}
\usepackage{amsmath}
\usepackage{color}
\usepackage{multirow}
\setcounter{tocdepth}{5}
\setcounter{secnumdepth}{4}
\definecolor{dkgreen}{rgb}{0,0.6,0}
\definecolor{gray}{rgb}{0.5,0.5,0.5}
\definecolor{mauve}{rgb}{0.58,0,0.82}
\definecolor{gray}{rgb}{0.4,0.4,0.4}
\definecolor{darkblue}{rgb}{0.0,0.0,0.6}
\titleformat{\paragraph}
{\normalfont\normalsize\bfseries}{\theparagraph}{1em}{}
\titlespacing*{\paragraph}
{0pt}{3.25ex plus 1ex minus .2ex}{1.5ex plus .2ex}
\renewcommand{\thesection}{\Roman{section}}
\hypersetup{
    colorlinks=true,
    linkcolor=black,
    filecolor=magenta,
    urlcolor=cyan,
}
\lstnewenvironment{cc}
{
\lstset{frame=tblr,
  language=C,
  aboveskip=3mm,
  belowskip=3mm,
  showstringspaces=false,
  columns=flexible,
  basicstyle={\small\ttfamily},
  numbers=none,
  numberstyle=\tiny\color{gray},
  keywordstyle=\color{blue},
  commentstyle=\color{dkgreen},
  stringstyle=\color{mauve},
  breaklines=true,
  breakatwhitespace=true,
  tabsize=3
}}
{}
\lstset{basicstyle=\ttfamily,
  showstringspaces=false,
  commentstyle=\color{red},
  keywordstyle=\color{blue}
}
\begin{document}
\title{
  \begin{minipage}\linewidth
      \centering
      \includegraphics[width=40mm]{resources/01.png}\vskip 20pt
      Rapport TP4 : Configuration complète d'un serveur
      \vskip 5pt
      \author{
        DRISSI Mohamed Reda \\
        \texttt{reda-mohamed@isty.uvsq.fr}
      }
    \end{minipage}
}
\maketitle
\newpage
\tableofcontents
\newpage
\section{Stratégie De Sauvegarde}
\subsection{Exercice 1}
La sauvegarde incrémentale permet d'appliquer les changements qui ont été effectués depuis la
dernière sauvegarde (uniquement les nouveaux fichiers, et les parties modifiés des fichiers déjà
présents). Puisque cette sauvegarde copie uniquement les changements, elle est considérablement
plus rapide que la sauvegarde complète. Le problème avec les sauvegardes incrémentales est que
la restauration est faible, puisqu'il faut restorer la dernière sauvegarde complète, puis
chaque sauvegarde incrémentale depuis, donc si nous avons une sauvegarde complète d'il y a un
mois puis des sauvegardes incrémentales quotidiennes, il faudra restaurer la sauvegarde complète
puis une trentaine de sauvegardes incrémentales, ce qui aussi bien évidemment implique des risques
de données dans le cas où une ou plusieurs sauvegardes sont corrompues.

\subsection{Exercice 2}
\begin{itemize}
  \item Une sauvegarde complète chaque mois qui sera compressé dans un .tar.bz2, qui commence à 19h
  \item Une sauvegarde differentielle chaque semaine qui commence à 19h chaque vendredi
  \item Une sauvegarde incrémentale toutes les 4 heures
\end{itemize}
Nous choisissons 19h parce que c'est le temps où la plupart si pas tout le monde a quitté son bureau.

\subsection{Exercice 3}

\begin{itemize}
  \item Base de données MariaDB : Nous allons utiliser mysqldump au lieu d'une simple copie pour
      éviter la corruption de transactions en cours. Notre base de données est presque vide, donc
      elle ne doit pas dépasser les 1Mb.
  \item Le répertoire \texttt{/home} est actuellement presque vide aussi, mais nous supposons un
        total de 3Gb
  \item Les données de l'application WordPress doivent être aux alentours de 60Mb au total :
      \begin{itemize}
        \item \texttt{/usr/share/wordpress/} 56Mb
        \item \texttt{/etc/wordpress} <1Mb
        \item \texttt{/var/lib/wordpress} <1Mb
      \end{itemize}
  \item Les données de l'annuaire LDAP contenus dans \texttt{/var/lib/ldap} qui eux aussi sont
  inférieures à 1Mb
\end{itemize}

\subsection{Exercice 4}
\begin{itemize}
  \item Bandes magnétiques:
    \begin{itemize}
      \item \textbf{Avantages} : Grosses capacités, très grande durée de vie.
      \item \textbf{Inconvénients} : Très lent, vulnérables aux aimans, même les plus faibles (ex
      hauts parleurs)
    \end{itemize}
  \item Disques durs magnétiques SATA (HDD):
    \begin{itemize}
      \item \textbf{Avantages}: Disponibles partout, pour de très faibles coûts, relativement fiables,
        compatibles partout.
      \item \textbf{Inconvénients}: Usure à cause des parties amovibles, vitesse très lente.
    \end{itemize}
    \begin{itemize}
  \item Stockages portables (hdd externes, clés usb)
      \item \textbf{Avantages}: facilité de transport et d'usage
      \item \textbf{Inconvénients}: capacité limitée, peu fiable, vitesse la plus lente après les
      bandes magnétiques.
    \end{itemize}
    \begin{itemize}
  \item Disques SSD SATA:
      \item \textbf{Avantages}: Beaucoup plus rapides que les HDDs, aucune partie amovible,
        température toujours basse.
      \item \textbf{Inconvénients}: Prix coûteux, durée de vie relative aux nombres de lectures/
      écritures
    \end{itemize}
    \begin{itemize}
  \item Disques SSD NVMe PCIe/M.2:
      \item \textbf{Avantages}: Beaucoup plus rapide que tout le reste
      \item \textbf{Inconvénients}: Beaucoup plus chers que tout le reste
    \end{itemize}
\end{itemize}
\subsection{Exercice 5}
Selon les supports, certains sont plus vulnérables à quelques conditions qu'à d'autres, et la
plupart sont vulnérables à ce qui nuit à l'informatique de manière générale; telle la température
élevée (ou trop basses), l'humidité, la poussière ou bien même les ondes électro-magnétiques.

\section{Espace de stockage}
\subsection{Exercice 6}
Nous avons déjà ajouter plusieurs disques durs sur virtualbox, ceci est trivial.
Nous choisissons un disque de 4Gb, qui dépasse largement notre taille de sauvegarde.
\subsection{Exercice 7}
Comme dans le TD2 nous allons utiliser parted ou fdisk ou cfdisk pour la création d'une partition.
Puis nous exécutons les commandes nécessaires au formatage et configuration de fstab:
\begin{tcolorbox}
  \begin{verbatim}
    $ mkfs.ext4 /dev/sdb1
    $ mkdir /backups
    $ echo " /dev/sdb1 /backups ext4 defaults 0 2" >> /etc/fstab
  \end{verbatim}
\end{tcolorbox}
\subsection{Exercice 8}
Garder un seul support de stockage lie nos données à la fiabilité de ce support. Utiliser une méthode
RAID pour une réplication de données ou un NAS, ou n'importe quelle autre méthode sera mieux adapté.

\section{Sauvegarde}

\subsection{Exercice 9}
Nous utilisons \texttt{tree} pour montrer l'arborescence
\begin{tcolorbox}
  \begin{verbatim}
    $ tree /backups
/backups
├── daily
│   ├── daily.0 <- dossier
│   └── daily.1 <- dossier
├── increments
├── monthly
│   └── monthly_2018-12-11.tar.bz2 <- archive
└── weekly
    └── weekly.0 <- dossier
  \end{verbatim}
\end{tcolorbox}
\newpage
\subsection{Exercice 10}
Nous évitons d'utiliser la varibale \texttt{\$PATH} parce que le script sera utilisé avec cron.
\begin{tcolorbox}
\begin{lstlisting}[language=bash]
#!/bin/sh
unset PATH
DAY=$(($(date +%u) - 1))
WEEK=$(($date +%w) - 1 ))
MONTH=$(($date +%m) - 1 ))
YEAR=$($date +%Y)
RSYNC="/usr/bin/rsync"
MKDIR="/bin/mkdir"
TAR="/bin/tar"
RM="/bin/rm"
SOURCE="/backups/include.txt" #currently contains /home
#can contain cache and other non important stuff
EXCLUDES="/backups/exclude.txt"
INCREMENTS="/backups/increments/$(date +%d-%m-%Y_%H:%M:%S)"
DEST="/backups/daily/daily.$DAY"
WEEKLY="/backups/weekly/weekly.$WEEK"
OPTS="-arf --ignore-erros --delete --files-from=$SOURCE\
      --exclude-from=$EXCLUDES"
DOPTS="--backup --backup-dir=$INCREMENTS --log-file=\
       /backups/log.daily.$(date+F%).log"
WOPTS="$WEEKLY --log-file=/backups/log.weekly.$(date+F%).log"
# Delete backups not needed
if [ $DAY -eq 0 ]; then
  # The following will only delete the folder if it's already full
  $MKDIR -p /backups/weekly/weekly."$WEEK" && $RM -r\
  /backups/weekly/weekly."$WEEK"/*
  $RSYNC $OPTS $WOPTS $DEST
  if [ $WEEK -eq 0]; then
    MONTHLY = /backups/monthly/year."$YEAR"/monthly."$MONTH"
    $MKDIR -p
    $TAR -cpjf "$MONTHLY"/monthly_$(date +F%).tar.bz2 /home
else
  # The following will only delete the folder if it's already full
  $MKDIR /backups/daily/daily."$DAY" && $RM -r\
  /backups/daily/daily."$DAY"/*
  $RSYNC $OPTS $DOPTS $DEST
\end{lstlisting}
\end{tcolorbox}
Nous pouvons aussi utiliser des utils comme \texttt{rsnapshot} ou \texttt{duplicity} pour
éviter toutes ces configurations.
\subsection{Exercice 11}
Nous aurons à répartir nos archives en de petits fichiers, la commande split sera parfaite pour
ceci.

\subsection{Exercice 12}
\begin{tcolorbox}
\begin{lstlisting}[language=bash]
mysqldump --all-databases --add-drop-database --add-drop-table \
--add-drop-trigger > /home/$USER/msqldump_$(date +%d-%m-%Y_%H:%M:%S).bak
\end{lstlisting}
\end{tcolorbox}
Ensuite elle sera sauvegardée avec le reste du home
\subsection{Exercice 13}
Si nous copions les fichiers d'une base de données alors qu'une transaction est en cours, nous
risquons de corrompre les données de cette DB.
Plusieurs solutions existent:
\begin{itemize}
  \item L'usage d'une sauvegarde logique avec un outil comme mysqldump.
  \item Verrouiller la base de données en lecture seule durant la sauvegarde.
  \item Arrêter la base de données lors de la sauvegarde.
  \item L'usage de la fonctionnalité snapshot si le système de fichiers le permets (zfs ou btrfs)
\end{itemize}
\subsection{Exercice 14}
La taille de la sauvegarde va causer une opération de sauvegarde très lente. Nous avons donc encore
plus de risque de corruption de cette base de données.\\
Les solutions que nous pouvons adopter:
\begin{itemize}
\item Usage de quelques outils plus rapides.
\item Verrouiller la base de données en lecture seule.
\item Usage de la fonctionnalité de snapshot, si possible.
\end{itemize}
\subsection{Exercice 15}
Ajouter les fichiers de ldap (/var/lib/ldap)dans le fichier "/backups/include.txt"
\subsection{Exercice 16}
Nous pouvons générer ces checksums avec la commande :
\begin{tcolorbox}
  \begin{verbatim}
find /path/to/backup -type f -exec md5sum {} + > /path/to/backup/checksum.md5
\end{verbatim}
\end{tcolorbox}
Nous pouvons les vérifier avec
\begin{tcolorbox}
  \begin{verbatim}
    md5sum -c /path/to/backup/checksum.md5
  \end{verbatim}
\end{tcolorbox}
\subsection{Exercice 17}
\begin{tcolorbox}
  \begin{verbatim}
    $ crontab -e
    * 0 * * * /path/to/script
  \end{verbatim}
\end{tcolorbox}
\section{Restauration}
\subsection{Exercice 18}
Génération d'un backup complet :
\begin{tcolorbox}
\begin{lstlisting}[language=bash]
    SOURCE="/home /var/lib/ldap \
            /usr/share/wordpress
            /etc/wordpress/
            /var/lib/wordpress/"
    DAY= $(($(date +%u) - 1 ))
    mkdir /backups/daily/daily.$DAY
    mysqldump --all-databases --add-drop-database --add-drop-table \
    --add-drop-trigger >/home/$USER/msqldump_$(date +%d-%m-%Y_%H:%M:%S).bak
    rsync -arf --ignore-erros --delete $SOURCE /backups/daily/daily.$DAY
\end{lstlisting}
\end{tcolorbox}
Génération d'un backup incrémental:
\begin{tcolorbox}
  \begin{verbatim}
    $ bash /root/backing_up.sh
  \end{verbatim}
\end{tcolorbox}
\subsection{Exercice 19}
Après avoir transféré les fichiers, nous commençons tout d'abord par bien vérifier
nos hashs md5 avec la commande :
\begin{tcolorbox}
  \begin{verbatim}
    $ md5sum -c /backups/daily/daily.1/checksum.md5
  \end{verbatim}
\end{tcolorbox}
Nous restaurons le backup:
\begin{tcolorbox}
  \begin{verbatim}
    DAY= $(($(date +%u) - 1 ))
    rsync -arf --ignore-erros --delete /backups/daily/daily.$DAY /
  \end{verbatim}
\end{tcolorbox}
Puis les données de notre base de données:
\begin{tcolorbox}
  \begin{verbatim}
    mysql > source /home/adminsys/msqldump_11-12-2018_19:48:35.bak
  \end{verbatim}
\end{tcolorbox}
\subsection{Exercice 20}
Pour se le faire nous allons faire la même chose que lorsqu'on restaure une sauvegarde normale,
le seul changement est de copier le répertoire htop-dev uniquement au lieu du backup complet.
\begin{tcolorbox}
  \begin{verbatim}
    rsync -arf --ignore-erros --delete \
    /backups/path/to/needed/backup/home/raj/htop-dev \
    /home/raj
  \end{verbatim}
\end{tcolorbox}
\end{document}
